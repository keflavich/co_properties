%\documentclass{article}
\documentclass[preprint]{aastex}
%\newcommand{\d}{\ensuremath{\textrm{d}}}
\newcommand{\ds}{\ensuremath{\textrm{d}s}}
\newcommand{\dv}{\ensuremath{\textrm{d}v}}
\newcommand{\dnu}{\ensuremath{\textrm{d}\nu}}
\newcommand{\kms}{\textrm{km~s}\ensuremath{^{-1}}}	%  km s-1
\usepackage{natbib}  % Requires natbib.sty, available from http://ads.harvard.edu/pubs/bibtex/astronat/
\usepackage{amsmath}
\citestyle{aa}  % (Author YYYY) references instead of (Author, YYYY)
%\bibliographystyle{/Users/adam/papers/latexfiles/apj_w_etal}
\begin{document}

\title{Derivation of the total column of a dipole molecule assuming LTE}

I derive the total columnn density of a dipole molecule assuming LTE and
$\tau<<1$, which are standard assumptions for many molecules, particularly CO
isotopologues (or is it isotopomers?  can never remember...) and $^{12}$CO in
outflow line wings.  I then show a few examples for CO.

This exercise has previously been performed for the CO 1-0 line in
\citet{bourke1997,garden1991,cabrit1990}, and probably many others, but to the
best of my knowledge has not been done for higher transitions.

\section{Radiative Transfer in a Dipole Line}
Equation of Radiative Transfer \citep[][eqn 1.9]{rohlfs}:
\begin{equation}
  \label{eqn:radtrans}
  \frac{dI_\nu}{\ds} = -\kappa_\nu I_\nu + \epsilon_\nu
\end{equation}

Definition of optical depth \citep[][eqn 1.15]{rohlfs}:
\begin{equation}
  \label{eqn:optdepth}
  d\tau_\nu = -\kappa_\nu \ds
\end{equation}

Spontaneous Emission \citep[][eqn 12.15]{rohlfs}:  $n_u$ is the volume density of molecules in the upper state.
\begin{equation}
  \epsilon_\nu = \frac{h \nu_{ul} A_{ul} n_u}{4 \pi} \varphi(\nu)
\end{equation}

Absorption:
\begin{equation}
  dE = -\frac{h \nu_{ul} B_{lu} n_l}{c} I_\nu~ \varphi(\nu)
\end{equation}

Stimulated Emission:
\begin{equation}
  dE = \frac{h \nu_{ul} B_{ul} n_u}{c} I_{\nu}~ \varphi(\nu)
\end{equation}

From \eqref{eqn:radtrans}, $\kappa_\nu$ should be the leading factor in front of $I_\nu$.
This is how \citet{rohlfs}  eqn 12.17 is derived.
\begin{equation}
  \label{eqn:kappa}
  \kappa_\nu = \frac{h \nu_{ul} B_{ul} n_u}{c} \varphi(\nu)
              -\frac{h \nu_{ul} B_{lu} n_l}{c} \varphi(\nu)
\end{equation}

The A and B values are related by
\begin{equation}
  \label{eqn:AB}
  A_{ul} = \frac{8\pi h \nu_{ul}^3}{c^3} B_{ul}
\end{equation}
and \begin{equation} B_{ul} = \frac{g_l}{g_u} B_{lu} \end{equation}

In LTE, Kirchoff's law is equal to the blackbody function \citep[][eqns 1.14, 12.8, 12.9]{rohlfs}
\begin{equation}
  \frac{\epsilon_\nu}{\kappa_\nu} = \frac{2 h \nu^3}{c^2} \left(\frac{g_u n_l}{g_l n_u} - 1 \right)^{-1}
                                  = \frac{2 h \nu^3}{c^2} \left[\exp\left(\frac{h\nu}{k_B T}\right) - 1\right]^{-1}
\end{equation}
\begin{equation}
  \label{eqn:distrib}
  \frac{n_u}{n_l} = \frac{g_u}{g_l} \exp\left(\frac{-h \nu_{ul} }{k_B T_{ex}}\right) 
\end{equation}

Plugging equations \eqref{eqn:distrib} and \eqref{eqn:AB} into \eqref{eqn:kappa} we get \citet{rohlfs} equation 12.17:
\begin{equation}
  \kappa_\nu = \frac{c^2}{8 \pi \nu_{ul}^2} \frac{g_u}{g_l} n_l A_{ul} \left[1-\exp\left(\frac{-h \nu_{ul} }{k_B T_{ex}}\right) \right] \varphi(\nu)
\end{equation}

However, that is not the most useful form of the equation, since we observe
\emph{emission}, which is the transition from the upper to lower state;
equation 12.17 makes sense theoretically because it is defining an absorption
coefficient.  This form is more useful (where we have replaced $n_l$ with $n_u$ using \eqref{eqn:distrib}):
\begin{equation}
  \kappa_\nu = \frac{c^2}{8 \pi \nu_{ul}^2} n_u A_{ul} \left[\exp\left(\frac{h \nu_{ul} }{k_B T_{ex}}\right) - 1 \right] \varphi(\nu)
\end{equation}

We then return to the optical depth \eqref{eqn:optdepth} and integrate over $\nu$:
\begin{equation}
  \int \tau_\nu \dnu = \frac{c^2}{8 \pi \nu_{ul}^2} A_{ul} \left[\exp\left(\frac{h \nu_{ul} }{k_B T_{ex}}\right) - 1 \right] \int \varphi(\nu) \dnu \int n_u \ds 
\end{equation}
\citep[$\int\varphi(\nu) \dnu \equiv 1$][eqn 12.1]{rohlfs}

Convert to column:
\begin{equation}
  \int n \ds = N
\end{equation}

Therefore the column of molecules in the upper state is
\begin{equation}
  N_u = \frac{8\pi \nu_{ul}^2}{c^2 A_{ul}} \left[\exp\left(\frac{h \nu_{ul} }{k_B T_{ex}}\right) - 1 \right]^{-1} \int \tau_\nu \dnu
\end{equation}

Convert from column to an observable, $T_B$ (or $T_B^*$ - that is up to the observer to fill in with correction factors etc, particularly
because CO tends to be extended, and other dipole molecules might as well).  Start with \citet{rohlfs} equation 15.29, which is the
conversion of brightness temperature to the full blackbody form times the thin-cloud component of the radiative transfer equation:
\begin{equation} 
  \label{eqn:tbrightnesscmb}
  T_B(\nu) = \frac{h \nu}{k_B} \left(\left[e^{h \nu / k_B T_{ex}} - 1\right]^{-1} - \left[e^{h \nu / k_B T_{CMB}} - 1\right]^{-1} \right) (1-e^{-\tau_\nu})
\end{equation}
Rearrange to solve for $\tau_\nu$:
\begin{equation}
  \tau_\nu = -\ln\left[ 1 - \frac{k_B T_B}{h \nu} \left(\left[e^{h \nu / k_B T_{ex}} - 1\right]^{-1} - \left[e^{h \nu / k_B T_{CMB}} - 1\right]^{-1} \right)^{-1} \right]
%  \int \tau_\nu \dnu = \int -\ln\left(1-\frac{T_B(\nu)}{T_{ex}}\right) \dnu 
%                     \approx_{\tau << 1} \int \frac{T_B(\nu)}{T_{ex}} \dnu 
%                     = \frac{\nu_{ul}}{c} \int \frac{T_B(v)}{T_{ex}} \dv
\end{equation}

The population of the upper state in terms of the observable $T_B$:
\begin{equation}
  N_u = \frac{8\pi \nu_{ul}^2}{c^2 A_{ul}} \left[\exp\left(\frac{h \nu_{ul} }{k_B T_{ex}}\right) - 1 \right]^{-1} \int -\ln\left[ 1 - \frac{k_B T_B}{h \nu_{ul}} \left(\left[e^{h \nu_{ul} / k_B T_{ex}} - 1\right]^{-1} - \left[e^{h \nu_{ul} / k_B T_{CMB}} - 1\right]^{-1} \right)^{-1} \right] \dnu
\end{equation}

Convert to \kms\ with $\dnu = \frac{\nu}{c} \dv$
\begin{equation}
  \label{eqn:nuppernoapprox}
  N_u = \frac{8\pi \nu_{ul}^3}{c^3 A_{ul}} \left[\exp\left(\frac{h \nu_{ul} }{k_B T_{ex}}\right) - 1 \right]^{-1} \int -\ln\left[ 1 - \frac{k_B T_B}{h \nu_{ul}} \left(\left[e^{h \nu_{ul} / k_B T_{ex}} - 1\right]^{-1} - \left[e^{h \nu_{ul} / k_B T_{CMB}} - 1\right]^{-1} \right)^{-1} \right] \dv
\end{equation}

Use the first term of the Taylor expansion: $\ln(1+x)\approx x-\frac{x^2}{2}+\frac{x^3}{3}\ldots$
\begin{equation}
  N_u = \frac{8\pi \nu_{ul}^3}{c^3 A_{ul}} \left[\exp\left(\frac{h \nu_{ul} }{k_B T_{ex}}\right) - 1 \right]^{-1} \int \frac{k_B T_B}{h \nu_{ul}} \left(\left[e^{h \nu_{ul} / k_B T_{ex}} - 1\right]^{-1} - \left[e^{h \nu_{ul} / k_B T_{CMB}} - 1\right]^{-1} \right)^{-1} \dv
\end{equation}

This can be simplified:
\begin{equation}
  \frac{ \left[e^{h \nu_{ul} /k_B T_{ex}} - 1 \right]^{-1} }{\left[e^{h \nu_{ul} / k_B T_{ex}} - 1\right]^{-1} - \left[e^{h \nu_{ul} / k_B T_{CMB}} - 1\right]^{-1} } = \frac{e^{h\nu_{ul}/k_B T_{CMB}} - 1}{e^{h\nu_{ul}/k_B T_{CMB}} - e^{h\nu_{ul}/k_B T_{ex}}}
\end{equation}

Which yields:
\begin{equation}
  \label{eqn:nupper}
  N_u = \frac{8\pi \nu_{ul}^2 k_B}{c^3 A_{ul} h }  \frac{e^{h\nu_{ul}/k_B T_{CMB}} - 1}{e^{h\nu_{ul}/k_B T_{CMB}} - e^{h\nu_{ul}/k_B T_{ex}}} \int T_B  \dv
\end{equation}

This can be converted to use $\mu$ instead of $A_{ul}$ using \citet{rohlfs} equation 15.20 $A_{ul}=\frac{64\pi^4}{3 h c^3}\nu^3 \mu_{ul}^2$:
\begin{equation}
  \label{eqn:nuppermu}
  n_u = \frac{3  }{8 \pi^3 \mu_e^2 } \frac{k_B}{\nu_{ul}} \frac{2 J_l + 3}{J_l+1} 
    \frac{e^{h\nu_{ul}/k_B T_{cmb}} - 1}{e^{h\nu_{ul}/k_B T_{cmb}} - e^{h\nu_{ul}/k_B T_{ex}}} \int T_B  \dv
\end{equation}

or in terms of $J_u$:
\begin{equation}
  \label{eqn:nuppermuju}
  N_u = \frac{3  }{8 \pi^3 \mu_e^2 } \frac{k_B}{\nu_{ul}} \frac{2 J_u + 1}{J_u} 
    \frac{e^{h\nu_{ul}/k_B T_{cmb}} - 1}{e^{h\nu_{ul}/k_B T_{CMB}} - e^{h\nu_{ul}/k_B T_{ex}}} \int T_B  \dv
\end{equation}

Stepping back, \eqref{eqn:tbrightnesscmb} can be simplified if one
assumes that the CMB contribution to the brightness temperature is negligible.
This assumption is very safe for high-lying transitions and high excitation
temperatures, only exceeding 5\% for $T_{ex} < 10$,6, and 4 K for the CO 1-0,
2-1, and 3-2 transitions (see figure \ref{fig:cmbcorrection}).
\begin{equation} 
  \label{eqn:tbrightnesscmb2}
  T_B(\nu) = \frac{h \nu}{k_B} \left[e^{h \nu / k_B T_{ex}} - 1\right]^{-1}  (1-e^{-\tau_\nu})
\end{equation}
Which can be solved for $\tau_\nu$ again:
\begin{equation}
  \tau_\nu = -\ln\left( 1 - \frac{k_B T_B}{h \nu} \left[e^{h \nu / k_B T_{ex}} - 1\right] \right) \approx \frac{k_B T_B}{h \nu} \left[e^{h \nu / k_B T_{ex}} - 1\right]
\end{equation}
This equation can be used following the above steps to derive a result in the form of \eqref{eqn:nupper}:
\begin{equation}
  \label{eqn:nupperapprox}
  N_u \approx \frac{8\pi \nu_{ul}^2 }{c^2 A_{ul} }  \frac{k_B}{h \nu_{ul}} \int T_B \dv
  = \frac{8\pi \nu_{ul} k_B}{c^3 A_{ul} h }  \int T_B \dv
  = \frac{3  k_B }{8 \pi^3 \mu_{ul}^2 \nu_{ul}^2 } \frac{2 J_l + 3}{J_l+1}   \int T_B \dv
\end{equation}


Note that \eqref{eqn:nuppernoapprox} is the full solution assuming LTE with \emph{no} approximations.


%\begin{equation}
%  \label{eqn:nupper}
%  N_u = \frac{8\pi \nu_{ul}^3}{c^3 A_{ul} T_{ex}} 
%  \left[\exp\left(\frac{h \nu_{ul} }{k_B T_{ex}}\right) - 1 \right]^{-1} \int T_B(v) \dv
%\end{equation}
%
%\begin{equation}
%  \label{eqn:nuppermu}
%  N_u = \frac{3 h }{8 \pi^3 \mu_e^2 } \frac{2 J_l + 3}{J_l+1}
%  \left[\exp\left(\frac{h \nu_{ul} }{k_B T_{ex}}\right) - 1 \right]^{-1} \int \frac{T_B(v)}{T_{ex}} \dv
%\end{equation}


\section{Partition Function}
We are interested in the total number (or column) of molecules, not that in a single state:
\begin{equation}
  n_{tot} = \sum_{J=0}^\infty n_J
\end{equation}

The Boltzmann distribution defines the distribution of molecular excitation states in LTE:
\begin{align}
  \frac{n_u}{n_l} &= \frac{g_u}{g_l} \exp\left(\frac{-E_{ul}}{k_B T_{ex}}\right) \label{eqn:Boltzmann}\\
  E_{ul} &= h B_e J_u(J_u+1) \label{eqn:eul} \\
  g_J &= 2J+1 \label{eqn:gj} 
\end{align}

By approximating the sum over all states as an integral, we can solve for the number in the 
ground state, which can in turn be used to solve for the number in any state:
\begin{align}
  n_{tot} & =        \sum_{J=0}^\infty n_J = n_0 \sum_{J=0}^\infty  (2J+1) \exp\left(-\frac{J(J+1) B_e h}{k_B T_{ex}}\right) \label{eqn:approxpartition}\\
          & \approx  \int_0^\infty  n_0  (2J+1) \exp\left(-\frac{J(J+1) B_e h}{k_B T_{ex}}\right) dJ \label{eqn:approxpartition2}\\
          & =        \left[ n_0 \frac{k_B T_{ex}}{B_e h} \exp\left(-\frac{J(J+1) B_e h}{k_B T_{ex}}\right) \right]^\infty_0 \label{eqn:approxpartition3}\\
          & =        n_0 \frac{k_B T_{ex}}{B_e h} \label{eqn:approxpartition4}
\end{align}
Rearrange to acquire 
\begin{equation}
  n_0 = n_{tot} \frac{B_e h}{k_B T_{ex}} 
\end{equation}

Plug back in to the Boltzmann distribution \eqref{eqn:Boltzmann}
\begin{equation}
  n_J = n_{tot} \frac{B_e h}{k_B T_{ex}} (2J+1) \exp\left(-\frac{J(J+1) B_e h}{k_B T_{ex}}\right)
\end{equation}
and finally rearrange
\begin{equation}
  \label{eqn:partition_final}
  n_{tot} = n_{J} \frac{k_B T_{ex}}{(2J+1)B_e h}  \exp\left(\frac{J(J+1) B_e h}{k_B T_{ex}}\right)
\end{equation}
Note that $n$ can be converted to $N$ by multiplying both sides by the path length.  

The numerical solution is acquired from plugging \eqref{eqn:Boltzmann} into \eqref{eqn:approxpartition}
\begin{equation}
  n_J = \left[ \sum_{j=0}^{j=j_{max}} (2j+1) \exp\left(-\frac{j(j+1) B_e h}{k_B T_{ex}}\right) \right] (2J+1) \exp\left(-\frac{J(J+1) B_e h}{k_B T_{ex}}\right)
\end{equation}

%Equation \eqref{eqn:nupper} can finally be plugged in to \eqref{eqn:partition_final} to
%acquire the column of the molecule as a function of the observed line temperature:
%\begin{equation}
%  \label{eqn:column}
%  N_{tot} = \frac{8\pi \nu_{ul}^3}{c^3 A_{ul}} \frac{k_B}{(2J+1)B_e h}  
%  \exp\left(\frac{J(J+1) B_e h}{k_B T_{ex}}\right)
%  \left[\exp\left(\frac{h \nu_{ul} }{k_B T_{ex}}\right) - 1 \right]^{-1} \int T_B(v) \dv
%\end{equation}

\begin{figure}
  \plotone{columnconversion_vs_tex}
  \caption{Using \eqref{eqn:nupper} (i.e. assuming $\tau_\nu << 1$) and a numerical treatment of 
  the partition function out to $J=200$, the conversion from $T_B(^{12}CO)$ to N(H$_2$) vs $T_{ex}$.  
  We assume X$_{CO} = 10^{-4}$ and LTE.}
\end{figure}

\begin{figure}
  \plotone{columnconversion_approximation}
  \caption{Comparison of the approximate to the exact (numerical) solutions of
  the partition function.  The approximate partition function is calculated
  using the approximation in \eqref{eqn:approxpartition}.  The exact
  (numerical) solution sums the population in all states out to $J=200$.}
\end{figure}

\begin{figure}
  \label{fig:cmbcorrection}
  \plotone{CMB_correction_factor}
  \caption{{\it Top:} Comparison of a full treatment of the optical depth
  including the contribution of the CMB as in \eqref{eqn:nuppernoapprox}
  with an approximation that ignores the CMB, \eqref{eqn:nupperapprox}.
  {\it Bottom:} Equation \eqref{eqn:nuppernoapprox}/Equation
  \eqref{eqn:nupperapprox}, which is the Correction Factor that needs to be
  applied to the approximate version to recover the full version.  Note that
  the correction factor is $<5\%$ for most reasonable excitation temperatures.  
  }
\end{figure}



%\section{Comparison to literature equations}
%\citet{arce2010} cites \citet{arce2001} which cites \citet{bourke1997} which cites \citet{garden1991} which
%starts from \citet{lang1980} equation 2-69...
%
%\citet{garden1991} appendix:
%\begin{equation}
%  N_{tot} = \frac{3k_B}{8\pi^3 B_e \mu^2} \frac{1}{J+1} \exp(\frac{h B_e J (J+1)}{k_B T_{ex}}) \frac{T_{ex} + hB_e / 3k_B}{1-\exp(-h nu/k_B T_{ex})} \int \tau_v dv
%\end{equation}
%where they have approximated the partition function $U=n_{tot}/n_0$ as
%\begin{eqnarray}
% U & =       & \sum_{J=0}^\infty g_J \exp\left(-\frac{h B_e J (J+1)}{k_B T_{ex}} \right) \\
%   & \approx & \frac{k_B}{h B_e} \left( \frac{T_{ex} + h B_e}{3k_B} \right) \\
%   & = & \frac{T_{ex}}{3 h B_e} + \frac{1}{3} \\
%\end{eqnarray}

\bibliographystyle{aasjournal}
\bibliography{column_derivation}
\end{document}
